%%%%%%%%%%%%%%%%%%%%%%%%%%%%%%%%%%%%%%%%%
% LaTeX Template
% Original author:
% Xavier Danaux (xdanaux@gmail.com) with modifications by:
% Ugo Turner
% License:
% CC BY-NC-SA 3.0 (http://creativecommons.org/licenses/by-nc-sa/3.0/)
%%%%%%%%%%%%%%%%%%%%%%%%%%%%%%%%%%%%%%%%%

%----------------------------------------------------------------------------------------
%	PACKAGES AND OTHER DOCUMENT CONFIGURATIONS
%----------------------------------------------------------------------------------------
\documentclass[11pt,a4paper,sans]{moderncv} % Font sizes: 10, 11, or 12; paper sizes: a4paper, letterpaper, a5paper, legalpaper, executivepaper or landscape; font families: sans or roman

\moderncvstyle{classic} % CV theme - options include: 'casual' (default), 'classic', 'oldstyle' and 'banking'
\moderncvcolor{grey} % CV color - options include: 'blue' (default), 'orange', 'green', 'red', 'purple', 'grey' and 'black'

\usepackage[scale=0.80]{geometry} % Reduce document margins
\usepackage[utf8]{inputenc}
\usepackage[T1]{fontenc}

%----------------------------------------------------------------------------------------
%	NAME AND CONTACT INFORMATION SECTION
%----------------------------------------------------------------------------------------

\firstname{Ugo} % Your first name
\familyname{TURNER} % Your last name

% All information in this block is optional, comment out any lines you don't need
\title{Développeur backend PHP}
% \address{123 Broadway}{City, State 12345}
\mobile{}
\email{}
\homepage{ugo-turner.com}{ugo-turner.com} % The first argument is the url for the clickable link, the second argument is the url displayed in the template - this allows special characters to be displayed such as the tilde in this example
% \extrainfo{additional information}
% \photo[70pt][0pt]{pictures/picture} % The first bracket is the picture height, the second is the thickness of the frame around the picture (0pt for no frame)
% \quote{"A witty and playful quotation" - John Smith}

%----------------------------------------------------------------------------------------

\begin{document}


%----------------------------------------------------------------------------------------
%	CURRICULUM VITAE
%----------------------------------------------------------------------------------------

\makecvtitle % Print the CV title

%----------------------------------------------------------------------------------------
%	WORK EXPERIENCE SECTION
%----------------------------------------------------------------------------------------

\section{Experiences}
\bigbreak

\cventry{01.2020\\12.2020}{Développeur Back-end Magento}{Absolunet}{\small{Montréal QC - 1 an}}{}{
  Absolunet est une agence E-commerce dans laquelle j'ai travaillé pour le compte de 3 sites à fort trafic/CA.
\newline{}
\begin{itemize}
\item Création et maintenance de modules Magento
\item Gestion des montées de version de la plateforme Magento (2.3.x à 2.4.x)
\item Communication clients (basecamp, zoom...)
\item Git, Docker, Bitbucket, Jira, Magento Cloud
\item Code review, poker planning, rétrospective
\end{itemize}}
\bigbreak

\cventry{04.2019\\10.2019}{Développeur Back-end PHP}{Softunik}{\small{Montréal QC - 6 mois}}{}{
  J'ai principalement travaillé sur l'évolution et la maintenance de Glow, une application
  de suivi de candidats éditée par Softunik.
\newline{}
\begin{itemize}
\item Évolution et maintenance applicative (framework interne en PHP 7.3)
\item Intégration de nouveaux clients (migration de leurs données dans Glow)
\item Création d'un outil de gestion de campagnes de recrutement utilisant l'API de Brockmeyer
\item Création d'un plugin Wordpress
\item Git, Gitlab, Jira
\end{itemize}}
\bigbreak

\cventry{01.2015\\12.2018}{Développeur web}{Hellocasa}{\small{Paris FR - 4 ans}}{}{
  Hellocasa est une entreprise française, spécialisée dans la vente et la gestion de travaux de bricolage, dépannage et rénovation.
  Arrivé dès le début du projet, mon rôle a été de concevoir de nombreuses fonctionnalités du site ainsi que des outils en interne.
  Début 2019, l'entreprise Hellocasa a été rachetée par EDF. Son nouveau nom est "IZI by EDF".
\newline{}
\begin{itemize}
\item Symfony 2.8 à 4.1, API-Platform
\item React.js/Redux
\item PhpUnit, Behat
\item DDD/CQRS, microservices
\item Git, Docker, Gitlab, Algolia, Kibana, Sonarqube
\item Code review, poker planning, rétrospective
\end{itemize}}
\bigbreak

\cventry{04.2014\\12.2014}{Stagiaire développeur web}{Bewoopi}{\small{Paris FR - 8 mois}}{}{
  Bewoopi est une agence spécialisée dans la création d'applications et le marketing mobile.
  Dans le cadre de son pôle R\&D, j'ai travaillé sur un prototype d'application capable de collecter des informations via les API de plusieurs réseaux sociaux.
\newline{}
\begin{itemize}
\item Python
\item PHP
\item Angular.js
\item BDD en graph (Neo4j)
\end{itemize}}
\bigbreak

\cventry{02.2012\\06.2012}{Stagiaire en gestion de projet E-commerce}{Tube2Com}{\small{Vannes FR - 4 mois}}{}{
  Agence de webmarketing et de SEO/SEA.
  Dans le cadre de ma licence professionnelle, j'ai effectué mon stage dans cette agence et avais pour mission d'entretenir, référencer et animer la communauté d'un site e-commerce de puériculture.
\newline{}
\begin{itemize}
\item Gestion d'un site e-commerce
\item SEO
\item Community management (newsletters, rédaction d'articles de blog et animation sur Twitter/Facebook)
\end{itemize}}
\bigbreak
\bigbreak
\bigbreak

%----------------------------------------------------------------------------------------
%	EDUCATION SECTION
%----------------------------------------------------------------------------------------

\section{Formations}
\bigbreak

\cventry{09.2012\\06.2014}{Master ingénierie de l’Internet (Maitrise)}{Université}{\small{Caen FR}}{}{PHP, Java, gestion de projet, systèmes et réseaux \bigbreak}
\cventry{09.2011\\06.2012}{Licence professionnelle e-commerce (Baccalauréat)}{IUT}{\small{Vannes FR}}{}{Gestion de projet, webmarketing, SEO, PHP \bigbreak}
\cventry{09.2009\\06.2011}{DUT techniques de commercialisation}{IUT}{\small{Angers FR}}{}{Marketing, stratégie d'entreprise, droit du e-commerce \bigbreak}


%----------------------------------------------------------------------------------------
%	TOOLS SECTION
%----------------------------------------------------------------------------------------

\section{Outils}
\bigbreak

Git, Gitlab, Vim, Docker, PHPStorm, GNU/Linux, Algolia, Kibana, Sonarqube, Metabase, Stripe, Jira


\end{document}
